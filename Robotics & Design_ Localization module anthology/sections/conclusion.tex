\section{Conclusion}

Throughout this project, we gained valuable hands-on experience in building a functional robotic system from the ground up. One of the most important lessons was the need to balance ambition with time management. Although we had a clear concept early on, the time constraints forced us to make adjustments and prioritize essential features. This highlighted the importance of starting integration early and making incremental progress rather than deferring complex tasks until the final stages.

We also learned the value of clear and constant communication within the team. Coordinating tasks, sharing updates, and adapting plans based on each member's progress made the difference between individual work and collaborative engineering. Dividing the workload logically (mechanics, electronics, code) helped us move forward efficiently, but it was equally important to stay synchronized.

Movement control also presented technical challenges. While omnidirectional movement offered great flexibility, it also introduced some complexities, particularly during initial testing and mechanical adjustments. We had to iterate several times to achieve reliable, smooth motion.

\subsection{Achievements}

The movement module successfully met our primary requirements:

\begin{itemize}
    \item \textbf{Omnidirectional mobility}: The three-wheel configuration with omnidirectional wheels allows the robot to move in any direction from any position, facilitating dynamic navigation in crowded spaces.
    
    \item \textbf{Basic obstacle detection}: The array of ultrasonic sensors provides adequate environmental awareness for autonomously avoiding collisions.
    
    \item \textbf{Modular design}: The final structure can easily integrate with other modules, thanks to its open architecture and clear electrical and mechanical interfaces.
    
    \item \textbf{Social presence}: The physical appearance successfully transforms a mechanical device into an engaging character that invites interaction.
\end{itemize}

\subsection{Future Work}

Several areas could be enhanced in future iterations:

\begin{itemize}
    \item \textbf{Improved localization}: Adding more sophisticated sensors or algorithms for more accurate position tracking would enhance navigation precision.
    
    \item \textbf{Advanced obstacle avoidance}: Implementing predictive algorithms to anticipate obstacles and plan smoother paths around them.
    
    \item \textbf{Battery performance optimization}: Refining power management to extend operational time between charges.
    
    \item \textbf{Enhanced human-robot interaction}: Expanding the communication capabilities beyond physical movement to include audio or visual feedback.
\end{itemize}

Overall, the project successfully delivered a functional movement module that serves as both a technical platform and an engaging social robot, demonstrating the effective integration of engineering and design principles.
