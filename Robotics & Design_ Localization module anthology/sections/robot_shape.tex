\subsection{Defining the Robot’s Shape: Square, Circular, Polygonal or Triangular Designs}

Following the decisions on movement type and wheel configuration, another crucial design consideration was the \textbf{shape} of the robot's base. The geometry of the robot directly affects not only its aesthetic but also its \textbf{manoeuvrability}, \textbf{stability}, \textbf{sensor placement} and \textbf{ability to navigate tight indoor spaces}. The primary shapes considered were \textbf{square}, \textbf{circular}, \textbf{polygonal} and \textbf{triangular} forms, each offering distinct advantages and challenges.

\begin{itemize}
    \item \textbf{Square:} Symmetry, ease of design, but corners can get caught and less smooth navigation.
    \item \textbf{Circular:} Excellent manoeuvrability, ideal for dynamic environments, but less space-efficient and more complex to construct.
    \item \textbf{Polygonal:} Balance between round and angular, unique structural advantages, but more complex to design and assemble.
    \item \textbf{Triangular:} Compact, simple three-wheel integration, but limited stability and challenging payload distribution.
\end{itemize}
