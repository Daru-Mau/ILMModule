\subsection{Strategy}

Our development strategy followed a bottom-up approach, starting with the physical structure and gradually integrating electronics and software. This allowed us to test each stage independently and ensure that all subsystems could function reliably before full integration.

The first step was to assemble the \textbf{structure} of the movement module in the Bovisa's \textit{Prototipi} lab. Using wood as the base material, we cut and mounted the two dodecagonal plates, positioning the vertical supports and verifying spacing and alignment based on our technical drawings. This gave us a solid physical base to work with and validated the dimensions from our CAD model.

Once the chassis was complete, we proceeded to \textbf{mount the electronic components} onto the structure. The three motors were installed and connected to their respective motor drivers, which were in turn wired to the microcontroller and power supply. The placement of components was adapted slightly during this stage to fit the real-world constraints of the wooden base, which differed slightly from the original digital model.

Finally, with the hardware in place, we began \textbf{programming the microcontroller}. We uploaded and tested the first version of the code, focusing on motor control and basic behavioral states for future expansion. This phase allowed us to validate our approach and confirm that the robot could move and behave according to the logic defined.
