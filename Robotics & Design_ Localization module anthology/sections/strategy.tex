\subsection{Strategy}

Our strategy during this phase focused on transforming abstract ideas into concrete design and implementation plans for the movement module of the robot. We followed a structured approach to ensure consistency across form, function, and technology.

\begin{enumerate}
    \item \textbf{Defining functionalities}
    
    We began by clearly outlining the movement module's responsibilities: patrolling the space, pausing at predefined locations to allow human interaction, and navigating autonomously to the charging station when battery levels are low. These core behaviors served as a foundation for every subsequent design decision.

    \item \textbf{Considering the physical form}
    
    We aimed to create a friendly and humorous appearance aligned with the social nature of the robot. The form of a cooking pot was chosen for its simplicity, roundness, and instantly recognizable shape, which also fits thematically with the microwave queue scenario.

    \item \textbf{Designing the 3D model of the movement module}
    
    Using 3D modeling tools, we created a rough prototype of the movement module's structure. This model focused on the chassis and physical arrangement of elements such as omnidirectional wheels, base support, and electronic housing. It helped us evaluate size constraints, stability, and potential mounting points for components.

    \item \textbf{Selecting electronic components}
    
    We identified the necessary electronic components, including three omnidirectional wheels, motor drivers, a microcontroller (ESP32), sensors for obstacle detection and battery monitoring, and components for the ticket dispensing system. We also defined the logical wiring and signal flow between them.

    \item \textbf{Beginning software development}
    
    In parallel, we started building the logic behind the movement module in code. This included route planning, obstacle avoidance, pausing logic, and low battery detection routines.
\end{enumerate}
