\subsection{Selecting the Sensing Technology: LiDAR, Ultrasonic, Camera and Other Options}

An essential step in developing the movement and localization module is selecting the \textbf{appropriate sensing technology}. The choice of sensors directly impacts the robot’s ability to perceive its environment, perform accurate localization, avoid obstacles and navigate efficiently indoors. Different types of sensors offer different strengths and weaknesses depending on the operating environment, required precision, cost and computational complexity.

\begin{itemize}
    \item \textbf{LiDAR:} High precision mapping, good range, strong performance in low light, but high cost and computational load.
    \item \textbf{Ultrasonic Sensors:} Low cost, simplicity, lightweight, but limited precision and susceptible to noise.
    \item \textbf{Cameras (RGB or Depth):} Rich environmental information, cost-effective, versatile, but lighting dependence and complex processing.
    \item \textbf{Other Options:}
    \begin{itemize}
        \item \textbf{IMU:} Provides orientation and movement data, but sensitive to drift.
        \item \textbf{Infrared Sensors:} Good for simple proximity detection, but limited in range and precision.
        \item \textbf{Magnetic Sensors:} Useful for specific localization systems using magnetic markers, but not practical for general indoor navigation.
    \end{itemize}
\end{itemize}
