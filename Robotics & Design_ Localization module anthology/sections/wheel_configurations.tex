\subsection{Defining the Number of Wheels: Three-Wheel vs. Four-Wheel Omnidirectional Configurations}

After selecting an omnidirectional movement strategy, the next major design challenge was to determine the optimal number of wheels for the robot. Specifically, we needed to choose between a \textbf{three-wheel} and a \textbf{four-wheel} omnidirectional configuration. This decision plays a critical role in the robot's stability, manoeuvrability, mechanical complexity and overall performance.

\textbf{Three-Wheel Omnidirectional Configuration}
\begin{itemize}
    \item \textbf{Advantages:} Simplicity, compact design, sufficient mobility.
    \item \textbf{Disadvantages:} Reduced stability, weight distribution challenges, limited redundancy.
\end{itemize}

\textbf{Four-Wheel Omnidirectional Configuration}
\begin{itemize}
    \item \textbf{Advantages:} Greater stability, improved traction, increased redundancy.
    \item \textbf{Disadvantages:} Higher mechanical complexity, larger footprint.
\end{itemize}
