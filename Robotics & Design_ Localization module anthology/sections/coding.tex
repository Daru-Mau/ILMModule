\subsection{Coding}

The code developed so far focuses on the core functionalities of the movement module: patrolling the environment, stopping periodically for interaction, and returning to a charging station when battery levels are low. The code is being developed in C++, using the Arduino framework, since it offers:

\begin{itemize}
    \item High compatibility with our chosen microcontroller (ATmega328P).
    \item Access to well-documented libraries for motor control, timers, and sensors.
    \item A simple structure that facilitates rapid prototyping and debugging.
\end{itemize}

The microcontroller is programmed directly via the Arduino IDE, allowing us to flash and test code iterations easily. The robot can switch between various behaviors such as:

\begin{itemize}
    \item Patrolling predefined points.
    \item Stopping and waiting for user interaction.
    \item Navigating to the charging station when the battery is low.
\end{itemize}

In this phase, we focused on:

\begin{itemize}
    \item Defining key states and transitions.
    \item Mapping out which pins control each motor and sensor.
    \item Testing the motors and sensors.
\end{itemize}

The code used for the testing of the HC-SR04 ultrasonic sensors, and the motors can be seen in the Appendix.
